\documentclass[12pt]{report}
\usepackage{natbib}
\usepackage{layout}
\usepackage[fleqn]{amsmath}
\usepackage{amssymb}
\usepackage{float}
\usepackage{graphicx}
\usepackage{caption}
\usepackage{subcaption}
\usepackage{geometry}
\usepackage{hyperref}
\usepackage{footnote}

\hypersetup{
    colorlinks   = true, %Colours links instead of ugly boxes
  urlcolor     = blue, %Colour for external hyperlinks
  linkcolor    = blue, %Colour of internal links
  citecolor   = blue %Colour of citations
}

% \newgeometry{textwidth=5in}
\makeatletter
\newcommand\footnoteref[1]{\protected@xdef\@thefnmark{\ref{#1}}\@footnotemark}
\makeatother

\begin{document}
\vspace{1.5in}
{\LARGE \textbf{TokenGen - Developer Documentation }}
\pagebreak
\newgeometry{top=2cm,bottom=2cm,left=1in,right=1in}


\section*{Design Overview} % {{{


The distributed TokenGen is implemented as a FastCGI extension to the
Apache web server. All requests at the Apache server is handled by the
FastCGI module. Additional context about the request via environment
variables is also passed on by Apache. The main thread in the FastCGI
module then computes the wait time for the request, and returns the
appropriate HTTP response to the client. The HTTP response contains the
meta HTTP “refresh” header, that automatically redirects the client to
the original web server  after the prescribed wait time. The redirect
URL contains the original URL that the client requested from the
application server, along with the following information embedded in
the URL string: the current time stamp, the wait time relative to the
current time stamp, and a HMAC token that is used to check the
authenticity of the reported wait time.

Each of the TokenGen replica reads from a configuration file the network (IP
and port) details of other replicas. Each replica has as many
threads as the number of peers replicas, each of which sends the wait
time array to the corresponding peer periodically. Apart from the
above the there are three running threads. One thread prints debug
logs. Another thread gathers capacity updates from the capacity
estimation module The main thread assigns wait times.

Every TokenGen instance has $n+3$ running threads. Where $n$ is the
number of peer proxies. The main thread is responsible for assigning
wait times to user requests. Second thread is a listening thread which
listens to all the inbound communication towards the TokenGen replica.
The third thread is used to print debug logs periodically. Inbound
communication contains of data from other peer TokenGen replicas and
capacity updates from the capacity estimation module. The remaining
$n$ threads communicate the current wait time data and current
incoming load with all other peer TokenGen replicas periodically.
Finally, one of the replicas runs the capacity estimation module that
provides the total server capacity estimates to all other replicas.



\section*{Source Code Overview - TokenGen} % {{{

The core functions and operations performed in the source files are
listed below

\textbf{functions.h}  contains helper function for various tasks in proxy1.c
parser.h - Reads the config file and populates various variables with
the correct values.  

\textbf{proxy1.c} 

\begin{itemize}
    \item
        readFromClient - This function reads
        incoming communications from TokenGen and CapacityEstimator. It is
        spawned in a different thread for each incoming connection and the
        data is accepted.Once the data transfer is over the thread is closed.
        This functions handles all the incoming requests weather it is from
        TokenGen or CapacityEstimator. It distinguishes the source based on
        the sentinel character sent by the client.  

    \item
        \verb create_server_socket - creates a server socket where incoming
        requests from TokenGen and CapacityEstimator arrives. Once the
        connection is accepted, this function calls the readFromClient
        function to get the data.

    \item
        writeToServer - n instances of this function run is parellel,
        where n is the number of peers. Once connected to a server( peer
        TokenGen) it will keep on sending the incoming rate and
        \verb visitor_count at specified time intervals in an infinite loop.

    \item
        main - core logic of TokenGen, we find the share of capacity 
        based on the data gathered by the above functions and redirect
        the requests to TokenCheck with the calculated wait time.
\end{itemize}


\textbf{queue.h} - Contains helper function written to maintain
a queue. not used currently.

\textbf{timer.h} - contains functions to write logs.

\restoregeometry
\end{document}
% vim: set spell:
